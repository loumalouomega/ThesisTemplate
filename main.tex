\documentclass[a4paper,10pt,leqno,graphicx,rotating,twoside]{book} % Book
\usepackage[DIV=14,BCOR=2mm,headinclude=true,footinclude=false]{typearea}
\usepackage{luacode}
\usepackage{graphicx,wrapfig,calc}
\newlength\myheight
\newlength\mydepth
\settototalheight\myheight{Xygp}
\settodepth\mydepth{Xygp}
\setlength\fboxsep{0pt}
\newcommand*\inlinegraphics[1]{%
  \settototalheight\myheight{Xygp}%
  \settodepth\mydepth{Xygp}%
  \raisebox{-\mydepth}{\includegraphics[height=\myheight]{#1}}%
}
\graphicspath{{./images/Color/}} % Color images
% \graphicspath{{./images/BW/}} % BW images
\setlength\intextsep{0pt}
\usepackage{float}
\usepackage{subfig}
\newsavebox{\measurebox}
\usepackage[english]{babel}
\usepackage{caption, subcaption}
\usepackage[parfill]{parskip}
%Options: Sonny, Lenny, Glenn, Conny, Rejne, Bjarne, Bjornstrup
\usepackage[Sonny]{fncychap}
\usepackage{amssymb}
\usepackage{mathtools}
\usepackage[sfdefault=cmbr]{isomath}
\usepackage{extsizes}
\usepackage{sidecap}

% NOTE: Here the different glossaries are declared
\usepackage[translate=false,acronym,nonumberlist,toc,xindy]{glossaries} %% NOT ENUMERATE
%\usepackage[translate=false,acronym,toc,xindy]{glossaries}  %% ENUMERATE
\setglossarysection{section}
\newglossary[tlg]{mathsymbols}{tld}{tdn}{Mathematical symbols}
\newglossary[tlg2]{otherterms}{tld2}{tdn2}{Other terms}

% Generate the glossary
\makeglossaries
\renewcommand{\glstextformat}[1]{\textbf{#1}}
\glsenablehyper
\usepackage[multiple]{footmisc}
\usepackage{numprint}
\usepackage{xcolor}
\usepackage{epigraph}
\usepackage[nottoc,numbib]{tocbibind}
\usepackage[bbgreekl]{mathbbol}
\usepackage{setspace}
\input{content/rgb}
%Highligh code
\usepackage{parcolumns}
\usepackage{textcomp}
%% NOTE: FOR CODE-Commented
% \usepackage{listings}
% \usepackage{minted}
\definecolor{dkgreen}{rgb}{0,0.6,0}
\definecolor{gray}{rgb}{0.5,0.5,0.5}
\definecolor{verylightgray}{gray}{0.85}

% Space between figures and text
\usepackage[belowskip=-15pt,aboveskip=0pt]{caption}
\setlength{\intextsep}{10pt plus 2pt minus 2pt}
\newcommand{\squeezeup}{\vspace{-2.5mm}}

\renewcommand{\theFancyVerbLine}{\sffamily \textcolor[rgb]{0.25,0.25,0.25}{\scriptsize \oldstylenums{\arabic{FancyVerbLine}}}}

% \newmintedfile[pythoncode]{python}{
%   % bgcolor=gray95,
%   % bgcolor=grey100,
%   fontfamily=helvetica, % tt, courier, helvetica
%   linenos=true,
%   fontsize=\scriptsize ,
%   numberblanklines=true,
%   numbersep=12pt,
%   numbersep=5pt,
%   gobble=0,
%   frame=leftline,
%   framerule=0.4pt,
%   framesep=2mm,
%   funcnamehighlighting=true,
%   tabsize=4,
%   obeytabs=false,
%   mathescape=false
%   samepage=false, %with this setting you can force the list to appear on the same page
%   showspaces=false,
%   showtabs =false,
%   texcl=false,
% }
%
% \newmintedfile[cppcode]{c++}{
%   % bgcolor=gray95,
%   % bgcolor=grey100,
%   fontfamily=helvetica, % tt, courier, helvetica
%   linenos=true,
%   fontsize=\scriptsize ,
%   numberblanklines=true,
%   numbersep=12pt,
%   numbersep=5pt,
%   gobble=0,
%   frame=leftline,
%   framerule=0.4pt,
%   framesep=2mm,
%   funcnamehighlighting=true,
%   tabsize=4,
%   obeytabs=false,
%   mathescape=false
%   samepage=false, %with this setting you can force the list to appear on the same page
%   showspaces=false,
%   showtabs =false,
%   texcl=false,
% }
%
% \newmintedfile[matlabcode]{matlab}{
%   % bgcolor=gray95,
%   % bgcolor=grey100,
%   fontfamily=helvetica, % tt, courier, helvetica
%   linenos=true,
%   fontsize=\scriptsize ,
%   numberblanklines=true,
%   numbersep=12pt,
%   numbersep=5pt,
%   gobble=0,
%   frame=leftline,
%   framerule=0.4pt,
%   framesep=2mm,
%   funcnamehighlighting=true,
%   tabsize=4,
%   obeytabs=false,
%   mathescape=false
%   samepage=false, %with this setting you can force the list to appear on the same page
%   showspaces=false,
%   showtabs =false,
%   texcl=false,
% }

%% NOTE: Decomment this in order to add a draft watermark
% % Use the following to make modification
% \usepackage[]{draftwatermark}
% %\SetWatermarkAngle{12}
% \SetWatermarkLightness{0.618}
% \SetWatermarkScale{1.2}
% \SetWatermarkText{DRAFT}

\usepackage{mathrsfs}
\usepackage{dsfont}
\usepackage[hyphens]{url}
\usepackage[bookmarks=true,breaklinks=true,colorlinks=true,allcolors=blue,linktocpage=true]{hyperref}
\usepackage{lastpage}
\usepackage{eurosym}
\usepackage{tikz,tikzscale,pgfplots,pgfplotstable,adjustbox}
% \pgfplotsset{compat=1.7}
\pgfplotsset{compat=newest,compat/show suggested version=false}
\usetikzlibrary{external}
% All externalized graphics go go the \externaldirectory
\tikzexternalize
\tikzsetexternalprefix{metafiles/build/}
% % Externalise only on-demand.
% \tikzexternaldisable
\usetikzlibrary{arrows.meta}
\usepackage{grffile}
\usetikzlibrary{plotmarks}
\usepgfplotslibrary{patchplots}
\tikzset{myarrow/.style={-{Triangle[length=3mm,width=1mm]}}}
\tikzset{%
  >={Latex[width=2mm,length=2mm]},
  % Specifications for style of nodes:
            base/.style = {rectangle, draw=black,
                           minimum width=4cm, minimum height=1cm,
                           text centered, font=\sffamily},
            transparent/.style = {rectangle, draw=white,
                           minimum width=1cm, minimum height=1cm,
                           text centered, font=\sffamily},
	  activityStarts/.style = {transparent, minimum width=0.25cm},
	  AD/.style = {transparent},
          process/.style = {base, minimum width=2.5cm, fill=white!15,
                           font=\ttfamily},
}
\usepackage{anysize} 
\marginsize{2cm}{2cm}{2cm}{2cm}
\usepackage[utf8x]{inputenc} % No compatible with biblatex
\usepackage{filecontents}

%Color in sectionig
\usepackage{sectsty}
\chapterfont{\color{DodgerBlue4}}  % sets colour of chapters
\sectionfont{\color{DodgerBlue4}}  % sets colour of sections
\subsectionfont{\color{DodgerBlue4}}  % sets colour of subsections
\subsubsectionfont{\color{DodgerBlue4}}  % sets colour of subsubsections

\usepackage[titletoc]{appendix} % When the type is a book

\usepackage{titlesec}
\titleclass{\subsubparagraph}{straight}[\subparagraph]
\newcounter{subsubparagraph}
\renewcommand{\thesubsubparagraph}{\Alph{subsubparagraph}}
\titleformat{\subsubparagraph}[runin]{\normalfont\normalsize\bfseries}{\thesubsubparagraph}{1em}{}
\titlespacing*{\subsubparagraph} {\parindent}{3.25ex plus 1ex minus .2ex}{1em}

%Bibliography
\usepackage[autostyle]{csquotes}% recommended in output (biblatex)

\newif\ifchapterbib

\chapterbibtrue

\ifchapterbib
        \usepackage[defernumbers=true,
                backend=biber,
%                 backend=bibtex8,
                backref=true,
%                 style=custom-numeric-comp, % Requiere descargar de Siarhei Khirevich's site
%                style=apa,
                style=alphabetic,
%                style=numeric-comp,
%                 style=ieee,
%                 style=natbib,
%                citestyle=authoryear,
                hyperref=true,
                natbib=true,
                citereset=chapter,
                maxcitenames=3,
                maxbibnames=100,
                refsection=chapter, %Bibliography for each chapter
                block=none]{biblatex}
    %% NOTE: Update correctly the folder
    \bibliography{/home/vicente/Trabajo/ThesisTemplate/content/bibliography.bib}
\else
    \usepackage[style=alphabetic,backend=biber]{biblatex}
    %% NOTE: Update correctly the folder
    \bibliography{/home/vicente/Trabajo/ThesisTemplate/content/bibliography.bib}
\fi

\usepackage[autostyle]{csquotes}% recommended in output (biblatex)
\usepackage[]{biblatex}

\DeclareBibliographyDriver{book}{%
  \printfield{title}%
  \newunit\newblock
  \printnames{author}%
  \newunit\newblock
  \printlist{publisher}%
  \newunit
  \printfield{edition}%
  \newunit
  \printlist{location}%
  \newunit
  \printfield{year}%
  \newunit
  \printfield{pages}%
  \finentry}
  
\DeclareBibliographyDriver{misc}{%
  \printfield{title}%
  \newunit\newblock
  \printnames{author}%
  \newunit\newblock
  \printlist{publisher}%
  \newunit
  \printfield{year}%
  \finentry}
  
\DeclareBibliographyDriver{thesis}{%
  \printfield{title}%
  \newunit\newblock
  \printnames{author}%
  \newunit\newblock
  \printlist{publisher}%
  \newunit
  \printfield{year}%
  \finentry}

\DeclareBibliographyDriver{article}{%
  \printfield{title}%
  \newunit\newblock
  \printnames{author}%
  \newunit\newblock
  \printlist{publisher}%
  \newunit
  \printlist{location}%
  \newunit
  \printfield{year}%
  \finentry}

\DeclareBibliographyDriver{manual}{%
  \printfield{title}%
  \newunit\newblock
  \printnames{author}%
  \newunit\newblock
  \printlist{publisher}%
  \newunit
  \printfield{url}%
  \finentry}
  
\DeclareBibliographyDriver{online}{%
  \printfield{title}%
  \newunit
  \printfield{url}%
  \finentry}

\DeclareFieldFormat{edition}%
                   {\ifinteger{#1}%
                    {\mkbibordedition{#1}\addthinspace{}ed.}%
                    {#1\isdot}}

%Extra item icons
\renewcommand{\labelitemi}{$\bullet$}
\renewcommand{\labelitemii}{$\circ$}
\renewcommand{\labelitemiii}{$\diamond$}
\renewcommand{\labelitemiv}{$\triangleright$}

%TOC sublevels
\setcounter{tocdepth}{5}
\setcounter{secnumdepth}{5}

%Font NOTE: Commented are different cool fonts, try them
\usepackage{avant}
% \usepackage[sfdefault]{universalis}
\usepackage[scaled]{helvet} % ss
%\usepackage{nimbus}
\usepackage[helvet]{sfmath} 
% \usepackage{cmbright}
\normalfont
\renewcommand*\familydefault{\sfdefault} %% Only if the base font of the document is to be sans serif
\usepackage[T1]{fontenc}
\usepackage{type1cm}
%\usepackage[lighttt]{lmodern}
% Using Courier font
\usepackage{courier}
\renewcommand{\ttdefault}{pcr}
\DeclareMathAlphabet\mathbfcal{OMS}{cmsy}{b}{n}

% Pseudo-code
\usepackage{algorithm}
\usepackage[noend]{algpseudocode}
\makeatletter
\def\BState{\State\hskip-\ALG@thistlm}
\makeatother
 
 %Head and foot page
\usepackage{fancyhdr}
\pagestyle{fancyplain}
% \renewcommand{\chaptermark}[1]{\markboth{\thechapter.\ #1}{}}
\renewcommand{\sectionmark}[1]{\markright{\thesection\ #1}}
% \rhead{}
% \lhead{}
\cfoot{}
\chead{}
\fancyhead[LE,RO]{\rightmark}
\fancyhead[LO,RE]{\leftmark}
\fancyfoot[RO,LE]{Page \hspace{0.3mm} \thepage \hspace{0.5mm} of \textbf{\pageref{LastPage}}}
\fancyfoot[LO,RE]{John Doe}
\renewcommand{\headrulewidth}{0.4pt}
\renewcommand{\footrulewidth}{0.4pt}
\renewcommand{\headwidth}{17cm} 
 
% Define commands to assure consistent treatment throughout document
\newcommand{\eqnref}[1]{(\ref{#1})}
\newcommand{\class}[1]{\texttt{#1}}
\newcommand{\package}[1]{\texttt{#1}}
\newcommand{\file}[1]{\texttt{#1}}
\newcommand{\BibTeX}{\textsc{Bib}\TeX}

%Metadata
\hypersetup{
    pdftoolbar=true,        % show Acrobat’s toolbar?
    pdfmenubar=true,        % show Acrobat’s menu?
    pdffitwindow=false,     % window fit to page when opened
    pdfnewwindow=true,      % links in new window
    pdftitle={PhD's thesis. Template},
    pdfauthor={John Doe},
    pdfcreator={John Doe},
    pdfproducer={John Doe},
    pdfsubject={Numeric},
    pdfkeywords={Numeric},
    bookmarksnumbered=true,     
    bookmarksopen=true,         
    bookmarksopenlevel=1,       
    colorlinks=true, % false: boxed links; true: colored links           
    pdfstartview=Fit,             
%     pdfstartview={FitH},    % fits the width of the page to the window
    pdfpagemode=UseOutlines,    % this is the option you were lookin for
    pdfpagelayout=TwoPageRight,
%     pdfpagelayout=TwoColumnRight,    
    linkcolor=RoyalBlue4,          % color of internal links (change box color with linkbordercolor)
    citecolor=DarkBlue,        % color of links to bibliography
    filecolor=green,      % color of file links
    urlcolor=blue,           % color of external links
} 
\pdfbookmark{\contentsname}{toc}

\usepackage{microtype} 

\usepackage{lipsum} %% NOTE: Only for template

\begin{document}

%% Acronyms
\newacronym{ode}{ODE}{Ordinary Differential Equation}
\newacronym{cfd}{CFD}{Computational Fluid Dynamics}
\newacronym{fd}{FD}{Finite Differences}
\newacronym{fsi}{FSI}{Fluid Structure Interaction}
\newacronym{fv}{FV}{Finite Volumes}
\newacronym{fvm}{FVM}{Finite Volume Method}
\newacronym{ss}{SS}{Small Strain}
\newacronym{fs}{FS}{Finite Strain}
\newacronym{fe}{FE}{Finite Element}
\newacronym{fem}{FEM}{Finite Element Method}
\newacronym{gfem}{GFEM}{Generalized Finite Element Method}
\newacronym{dem}{DEM}{Discrete Element Method}
\newacronym{cad}{CAD}{Computer-Aided Design}
\newacronym{pde}{PDE}{Partial Differential Equation}
\newacronym{fd}{FD}{Finite Difference}
\newacronym{fea}{FEA}{Finite Element Analysis}
\newacronym{cam}{CAM}{Computer Aided Manufacturing}
\newacronym{cad}{CAD}{Computer Aided Design}
\newacronym{cae}{CAE}{Computer Aided Engineering}

%% Glossary
% Math symbols
\newglossaryentry{pi}
{
    name={\ensuremath{\pi}},
    type=mathsymbols,
    description={Ratio of circumference of circle to its diameter},
    sort=pi
}

\newglossaryentry{kappa}
{
    name={\ensuremath{\kappa}},
    type=mathsymbols,
    description={The condition number of a function measures how much the output value of the function can change for a small change in the input argument. A problem with a low condition number is said to be \textbf{well-conditioned}, while a problem with a high condition number is said to be \textbf{ill-conditioned}},
    sort=kappa
}

% Other terms
\newglossaryentry{C++}
{
    name={\textit{C++}},
    type=otherterms,
    description={It is a general-purpose programming language created as an extension of the C programming language, but \gls{oop} designed},
    sort=LangCpp
}

\newglossaryentry{python}
{
    name={\textit{Python}},
    type=otherterms,
    description={It is an interpreted, high-level, general-purpose, \gls{oop} scripting programming language},
    sort=LangCpp
}
%
\newglossaryentry{openmp}
{
    name={\textit{OpenMP}},
    type=otherterms,
    description={\textit{Open Multi-Processing} is an \gls{api} that supports multi-platform shared memory multiprocessing programming},
    sort=libopenmp
}


\begin{titlepage}
    \begin{center}
    \begin{tabular}{ccc}
  \includegraphics[width=4cm,height=6cm,keepaspectratio]{logos/Logo_UPC.pdf} & \hspace{8cm}
& \includegraphics[width=4cm,height=6cm,keepaspectratio]{logos/logo_ETSECCPB.jpeg} \\
    \end{tabular}
    \vfill
    \LARGE{Universitat Politècnica de Catalunya-ETSECCPB} \\
    \Large{Programa de doctorado en Ingenieria Civil} \\
    \Huge{\textbf{PhD's thesis}}
    \vfill
    \Huge{Template thesis}
    \vfill
    \Large{John Doe}
    \vfill
    \large{Director: Name}
    \\
    \large{Co-director: Second Name}
    \vfill
%     \begin{figure}[h]
%     \begin{center}
%     \includegraphics[width=0.4\textwidth]{introduction/signature.pdf}
%     \includegraphics[width=0.4\textwidth]{introduction/signature2.pdf}
%     \end{center}
%     \end{figure}
%     \vfill
%     \Large{\today}
  \end{center}
\end{titlepage}

\cleardoublepage
\pdfbookmark{\contentsname}{toc}

\chapter*{}

\vspace{7.0cm}

\large

\inlinegraphics{quote1.pdf} \emph{Ce qui est simple est toujours faux. Ce qui ne l’est pas est inutilisable} \hspace{0.635\textwidth} \inlinegraphics{quote2.pdf}

\normalsize

\rule[0pt]{0.75\textwidth}{0.4pt}

\emph{Paul Valéry (1871 – 1945 AD, French poet, essayist, and philosopher)}

\newpage

\textcolor{white}{Empty space}

\vspace{10cm}

\large

\inlinegraphics{quote1.pdf} \emph{Essentially, all models are wrong, but some are useful} \hspace{0.51\textwidth} \inlinegraphics{quote2.pdf}

\normalsize

\rule[0pt]{0.675\textwidth}{0.4pt}

\emph{Aphorism}, attributed to \emph{George E. P. Box (1919 – 2013 AD, British statistician)}

\chapter*{}
\begin{center}
  \LARGE \textbf{Abstract} \normalsize
\end{center}
\begin{quotation}
  \noindent \input{content/introduction/abstract}
\end{quotation}
\clearpage
\begin{center}
  \LARGE \textbf{Resumen} \normalsize
\end{center}
\begin{quotation}
  \noindent \input{content/introduction/resumen}
\end{quotation}
\clearpage

%%-----------Table of Contents------------------
\tableofcontents{}

\newpage

\newenvironment{acknowledgements}%
    {\cleardoublepage\thispagestyle{empty}\null\vfill\begin{center}%
    \bfseries Acknowledgements\end{center}}%
    {\vfill\null}
        \begin{acknowledgements}
        \lipsum[2-4]

        \end{acknowledgements}


\newpage

\part{State of art}

\input{content/introduction/introduction}

\part{Conclusions}

\chapter{Final conclusions}\label{cap:finalconclusions} 

\epigraph{\includegraphics[scale=0.1]{quote1.pdf} \emph{Both if you think you can do it and if you don’t, you are in the right} \\ \includegraphics[scale=0.1]{quote2.pdf}}{\emph{Henry T. Ford \\ (1863 - 1947 AD, Engineer, Industrialist and Philanthropist)}}

\section{Introduction}

\lipsum

\ifchapterbib
    % \printbibliography[heading=subbibliography]

\addcontentsline{toc}{section}{Bibliography of chapter} %Bibliografia para el capítulo
\printbibheading
\begingroup
\raggedright
\sloppy
%\newrefcontext[labelprefix={Book-}]
\newrefcontext[labelprefix={Book}]
\printbibliography[type=book,heading=subbibliography,resetnumbers=true,title={Books}]
%\newrefcontext[labelprefix={Art-}]
\newrefcontext[labelprefix={Art}]
\printbibliography[type=article,heading=subbibliography,resetnumbers=true,title={Articles}]
%\newrefcontext[labelprefix={Man-}]
\newrefcontext[labelprefix={Man}]
\printbibliography[type=manual,heading=subbibliography,resetnumbers=true,title={Manuals}]
%\newrefcontext[labelprefix={Mas-}]
\newrefcontext[labelprefix={Mas}]
\printbibliography[type=misc,heading=subbibliography,resetnumbers=true,title={Master's thesis.}]
%\newrefcontext[labelprefix={PhD-}]
\newrefcontext[labelprefix={PhD}]
\printbibliography[type=thesis,heading=subbibliography,resetnumbers=true,title={Ph.D.'s thesis}]
%\newrefcontext[labelprefix={Onl-}]
\newrefcontext[labelprefix={Onl}]
\printbibliography[type=online,heading=subbibliography,resetnumbers=true,title={Online resources}]
\endgroup

\fi


\chapter{Future works}\label{cap:futureworks} 

In this chapter we present the future developments that will follow to this PhD' thesis.

\begin{itemize}

\lipsum
 
\end{itemize}

%\ifchapterbib
%    % \printbibliography[heading=subbibliography]

\addcontentsline{toc}{section}{Bibliography of chapter} %Bibliografia para el capítulo
\printbibheading
\begingroup
\raggedright
\sloppy
%\newrefcontext[labelprefix={Book-}]
\newrefcontext[labelprefix={Book}]
\printbibliography[type=book,heading=subbibliography,resetnumbers=true,title={Books}]
%\newrefcontext[labelprefix={Art-}]
\newrefcontext[labelprefix={Art}]
\printbibliography[type=article,heading=subbibliography,resetnumbers=true,title={Articles}]
%\newrefcontext[labelprefix={Man-}]
\newrefcontext[labelprefix={Man}]
\printbibliography[type=manual,heading=subbibliography,resetnumbers=true,title={Manuals}]
%\newrefcontext[labelprefix={Mas-}]
\newrefcontext[labelprefix={Mas}]
\printbibliography[type=misc,heading=subbibliography,resetnumbers=true,title={Master's thesis.}]
%\newrefcontext[labelprefix={PhD-}]
\newrefcontext[labelprefix={PhD}]
\printbibliography[type=thesis,heading=subbibliography,resetnumbers=true,title={Ph.D.'s thesis}]
%\newrefcontext[labelprefix={Onl-}]
\newrefcontext[labelprefix={Onl}]
\printbibliography[type=online,heading=subbibliography,resetnumbers=true,title={Online resources}]
\endgroup

%\fi


\addcontentsline{toc}{part}{Appendices}
\appendix
\appendixpage
\noappendicestocpagenum
\addappheadtotoc

\input{content/appendix/appen}

%%------------List of Tables----------------------
\listoftables{}
%%------------List of Figures----------------------
\listoffigures{}

%Print the glossary
\chapter*{Glossaries}
\addcontentsline{toc}{chapter}{Glossaries}
\glsaddall
\printglossaries
% \printglossary[type=\acronymtype]
% \printglossary
% \printnoidxglossary[type=acronym,sort=letter]
% \printnoidxglossary[type=symbols,sort=letter]

\end{document}
